\documentclass[12pt,oneside,a4paper,bibtotoc,liststotoc]{scrreprt}
\usepackage{remreset}

%% Listings
\usepackage{listings}
\lstset{
  basicstyle=\footnotesize, 
  stringstyle=\ttfamily,
  frame=single,
  numberbychapter=false, % Listings vom Anfang bis zum Ende
                         % durchnummerieren
  captionpos=b
}
% Fußnoten, Tabellen und Grafiken von Anfang bis Ende durchnummerieren.
\usepackage{chngcntr}
\counterwithout{footnote}{chapter}
\counterwithout{table}{chapter}
\counterwithout{figure}{chapter}

% Überschriften bis zur dritten Ebene durchnummerieren
\setcounter{secnumdepth}{3}
% Überschriften bis zur zweiten Ebene in den TOC
\setcounter{tocdepth}{2}

%% Typographie-Schnickschnack
\usepackage[T1]{fontenc}
\usepackage{lmodern}
\usepackage{ellipsis} 

% Schönere Zeilenumbrüche
\usepackage{microtype}
\tolerance 1414
\hbadness 1414
\emergencystretch 1.5em
\hfuzz 0.3pt
\vfuzz \hfuzz
\raggedbottom
% Schusterjungen und Hurenkinder unterdrücken
\clubpenalty = 10000 
\widowpenalty = 10000 
\displaywidowpenalty = 10000 

\usepackage[utf8]{inputenc}
% Mathematischen Formelsatz erlauben
\usepackage{amsmath}
\usepackage{amssymb}
\usepackage[ngerman]{babel}

%% Wenn ein Index verwendet werden soll, folgendes auskommentieren
% \usepackage{makeidx}
% \makeindex

\usepackage{graphicx}

%% Literaturverzeichnis
\usepackage{jurabib}
\jurabibsetup{
	authorformat=abbrv,
        titleformat={short,commasep},
	commabeforerest,
	see,
        square,
}
% Typographischer Schnickschnack für Bibliographie
\renewcommand*{\jbauthorfont}{\textsc}
\renewcommand*{\biblnfont}{\scshape\textbf}
\renewcommand*{\bibfnfont}{\normalfont\textbf}

% Referenzen verlinken
\usepackage{hyperref}
\hypersetup{ 
    colorlinks,
    citecolor=black,
    filecolor=black,
    linkcolor=black,
    urlcolor=black 
}

% Grad-Zeichen definieren
\newcommand{\grad}{\mbox{\(\mathsurround=0pt{}^\circ\)}}
% (r)-Zeichen definieren
\def\TReg{\textsuperscript{\textregistered}}
% z.B., d.h. usw. Typographisch korrekt setzen
\usepackage{xspace}
\newcommand{\zB}{\mbox{z.\,B.}\xspace}
\newcommand{\dH}{\mbox{d.\,h.}\xspace}
\newcommand{\ua}{\mbox{u.\,a.}\xspace}
\newcommand{\so}{\mbox{s.\,o.}\xspace}

%% Schönere Tabellen
\usepackage{booktabs}
\usepackage{tabularx}

%% Ränder setzen
\usepackage[left=5cm,right=3cm,top=1.5cm,bottom=2cm,includeheadfoot]{geometry}

%% Abkürzungsverzeichnis
\usepackage[printonlyused]{acronym}

%% Allow really fixed positions for Tables
\usepackage{float}
\restylefloat{table}

%% Gedrehte Tabellen
\usepackage{rotating}

%% Zeilenabstand 1.5. Nur setzen, wenn erforderlich, sieht ziemlich
%% hässlich aus.
% \usepackage{setspace} 
%% 150%
% \onehalfspacing % Zeilenabstand
%% 125%
%\linespread{1.25}
%% 110%
%\linespread{1.1}
%% 105%
%\linespread{1.05}
%% 102,5%
%\linespread{1.025}

%% Trennung tweaken: Manche Wörter will man ggf. anders trennen als
%% der Algorithmus das vorsieht oder ggf. auch garnicht
\hyphenation{Java-Script}
\hyphenation{Cookies}

\begin{document}

\begin{titlepage}

\includegraphics[width=8cm]{img/fhk.jpg}

\begin{center}
\textbf{\LARGE Fachhochschule Köln\\
Cologne University of Applied Sciences\\[0.5cm]
Campus Gummersbach\\[0.5cm]
Fakultät für Informatik und Ingenieurwissenschaften\\[1cm]}
 
\Large Verbundstudiengang Wirtschaftsinformatik\\[1cm]

\Large Template \\[1cm]
  
% Title
{ \huge \bfseries \LaTeX-Template für Thesen an der FH Köln}\\

\vfill

\begin{table}[h]
\centering
\begin{tabular}{p{4cm}p{5cm}}
  Prüfer:         & Prof. Dr. Max Mustermann \\
  Zweitprüfer:    & Prof. Dr. Monika Mustermann \\
                  &  \\
  vorgelegt am:   & \today \\
  von:            & Stephan Student \\
  aus:	  	  & Gummersbach \\
  Telefon-Nr.:	  & +49-123-456789 \\
  Matrikel-Nr.:   & 123456789 \\
  E-Mail-Adresse: & student@fh-koeln.de
\end{tabular}
\end{table}
\end{center}
\end{titlepage}

\tableofcontents

\chapter{Einleitung}

Dieses \LaTeX Template dient als Vorlage für Bachelor- und
Masterthesen, Haus-, Seminar- und Diplomarbeiten an der \ac{FH} Köln.

Zitate im Text so wie \cite[vgl.][S.~1]{Lamport} sind mittels
\texttt{\textbackslash cite}, Zitate als Fußnote, zum
Beispiel\footcite[vgl.][S.~1]{Lamport} mittels \texttt{\textbackslash
  footcite} einfügbar. Die \ac{FH} Köln hat zwar eine Empfehlung zum
Zitierstil hin zu Zitaten mittels Fußnote, letztendlich ist es aber
eine Absprache mit dem Betreuer der Arbeit.

\chapter{Textgliederung und Verweise}

Im folgenden sieht man die -- per default -- vorgesehenen
Gliederungsmöglichkeiten. In der Präambel des Dokumentes wird
gesteuert, bis zu welcher Tiefe die Ebenen nummeriert werden bzw. im
Inhaltsverzeichnis auftauchen sollen:

\begin{lstlisting}[language=TeX,caption=Steuerung der
  Nummerierungstiefe und Inhaltsverzeichnistiefe in der
  Prämbel,label=praeambel_tiefe]
% Ueberschriften bis zur dritten Ebene durchnummerieren
\setcounter{secnumdepth}{3}
% Ueberschriften bis zur zweiten Ebene in den TOC
\setcounter{tocdepth}{2}
\end{lstlisting}

\section{Verweise in \LaTeX}

Dies ist ein Abschnitt. Von hier aus kann man \zB sehr bequem auf den
Unterabschnitt \glqq \nameref{dritter_beispielunterabschnitt}\grqq\
mit der Nummer \ref{dritter_beispielunterabschnitt} auf
S.~\pageref{dritter_beispielunterabschnitt} verweisen. Das geht
übrigens auch bequem mit Listings wie dem eben gezeigten
Listing~\ref{praeambel_tiefe}, dem wir die Beschriftung \glqq
\nameref{praeambel_tiefe}\grqq\ gegeben haben, und das sich auf auf
S.~\pageref{praeambel_tiefe} befindet. Wir können das Listing ebenso
bequem referenzieren können wie den Unterabschnitts vorhin. Ändern
sich Seite, Beschriftung, fortlaufende Nummern usw., ändern sich die
entsprechenden Bezeichner im Text natürlich gleich mit. Nach dem
gleichen Schema kann alles, was mit einem Label versehen werden kann,
referenziert werden, also \zB Abschnitte, Abbildungen, Tabellen,
Listings oder auch beliebige Stellen im Text.

\subsection{Erster Beispielunterabschnitt}

Dies ist ein Unterabschnitt.

\subsubsection{Erster Unterunterabschnitt}

Dies ist ein Unterunterabschnitt.

\subsubsection{Zweiter Unterunterabschnitt}

Dies ist ein Unterunterabschnitt.

\paragraph{Erster Absatz} Dies ist ein Absatz.

\paragraph{Zweiter Absatz} Dies ist noch ein Absatz.

\subsection{Dritter Beispielunterabschnitt}
\label{dritter_beispielunterabschnitt}

Dies ist ein Unterabschnitt.

\chapter{Abbildungen und Tabellen}

Abbildungen, Tabellen und andere Objekte lassen sich leicht in \LaTeX
einbinden und referenzieren.

\section{Abbildungen}

Abbildungen lassen sich in \LaTeX sehr einfach einbinden:

\begin{figure}[H]
  \begin{centering}
    \includegraphics[width=0.8\textwidth]{img/example.png}
    \caption{Beispielbild}
    \label{example_image}
  \end{centering}
\end{figure}

\section{Tabellen}

Tabellen lassen sich in \LaTeX extrem vielseitig gestalten. Ein
Beispiel für eine einfache Tabelle findet sich in
Tabelle~\ref{example_table}. Komplexere Beispiele finden sich \zB in
\cite[][]{WikibookTables}.

\begin{table}
  \centering
  \begin{tabular}{l|ll}
    \toprule
    Überschrift & Überschrift & Überschrift \\
    \midrule
    Zelle & Zelle & Zelle \\
    Zelle & Zelle & Zelle \\
    Zelle & Zelle & Zelle \\
    Zelle & Zelle & Zelle \\
    \bottomrule
  \end{tabular}
  \caption{Beispieltabelle}
  \label{example_table}
\end{table}

Tabellen können mittels \texttt{sidewaystable} auch quer zur Seite
gerendert werden:

\begin{sidewaystable}
  \centering
  \begin{tabular}{llll}
    \toprule
    Head 1 & Head 2 & Head 3 & Head 4 \\
    \midrule
    Row 1-1 & Row 2-1 & Row 3-1 & Row 4-1 \\
    Row 1-2 & Row 2-2 & Row 3-2 & Row 4-2 \\
    Row 1-3 & Row 2-3 & Row 3-3 & Row 4-3 \\
    \bottomrule
  \end{tabular}
  \caption{Beispiel für eine Quertabelle}
  \label{example_table2}
\end{sidewaystable}

\section{Referenzen}

Bei Abbildungen gibt es (wie bei Tabellen, Listings usw.) die
Möglichkeit, die Positionierung festzulegen. Normalerweise \glqq
floaten\grqq\ Abbildungen und Tabellen in \LaTeX, \dH sie werden dort
im Text untergebracht, wo sie am besten aussehen, nicht so wie im
Quelltext positioniert sind. Das erfordert natürlich eine andere Art
der Referenzierung von Abbildungen, als die meisten das gewohnt
sind. Die Doppelpunktnotation (\so) funktioniert dann natürlich nicht,
sondern man müsste \zB etwas wie \glqq Abbildung~\ref{example_image}
ist ein Beispiel für die Einbindung von Abbildungen in \LaTeX \grqq\
schreiben. Tabelle~\ref{example_table} ist \zB ein floatendes Objekt
und wird so platziert, dass sich ein möglichst harmonisches
Seitenlayout ergibt.

Das Floating lässt sich jedoch steuern. Eine umfangreiche Anleitung
dazu findet sich in \cite[][]{WikibookFloats}. Ein extremes Beispiel
ist Abbildung~\ref{example_image} die fest an die Position hinter dem
Doppelpunkt gezwungen wird.

\chapter*{Abkürzungsverzeichnis}
\addcontentsline{toc}{chapter}{Abkürzungsverzeichnis}
\begin{acronym}[Acronms]
\acro{FH}{Fachhochschule Köln}

\end{acronym}

\bibliography{thesis}{}
\bibliographystyle{jureco} %% jurabib, jureco, jurunsrt, jox

\listoffigures

\listoftables

\lstlistoflistings

%% \printindex

\appendix

\chapter{Erklärung}

Ich versichere, die von mir vorgelegte Arbeit selbständig verfasst zu
haben. Alle Stellen, die wörtlich oder sinngemäß aus veröffentlichten
oder nicht veröffentlichten Arbeiten anderer entnommen sind, habe ich
als entnommen kenntlich gemacht. Sämtliche Quellen und Hilfsmittel,
die ich für die Arbeit benutzt habe, sind angegeben. Die Arbeit hat
mit gleichem Inhalt bzw. in wesentlichen Teilen noch keiner anderen
Prüfungsbehörde vorgelegen.

\bigskip

Gummersbach, den \today

\bigskip

\bigskip

\bigskip

\bigskip

\bigskip

\bigskip

(Unterschrift)

\end{document}

%% end of file
