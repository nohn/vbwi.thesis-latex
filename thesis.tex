\documentclass[12pt,oneside,a4paper,bibtotoc,liststotoc]{scrreprt}
\usepackage{remreset}

%% Listings
\usepackage{listings}
\lstset{
  basicstyle=\footnotesize, 
  stringstyle=\ttfamily,
  frame=single,
  numberbychapter=false, % Listings vom Anfang bis zum Ende
                         % durchnummerieren
  captionpos=b
}
% Fußnoten, Tabellen und Grafiken von Anfang bis Ende durchnummerieren.
\usepackage{chngcntr}
\counterwithout{footnote}{chapter}
\counterwithout{table}{chapter}
\counterwithout{figure}{chapter}

% Überschriften bis zur dritten Ebene durchnummerieren
\setcounter{secnumdepth}{3}
% Überschriften bis zur zweiten Ebene in den TOC
\setcounter{tocdepth}{2}

%% Typographie-Schnickschnack
\usepackage[T1]{fontenc}
\usepackage{lmodern}
\usepackage{ellipsis} 

% Schönere Zeilenumbrüche
\usepackage{microtype}
\tolerance 1414
\hbadness 1414
\emergencystretch 1.5em
\hfuzz 0.3pt
\vfuzz \hfuzz
\raggedbottom
% Schusterjungen und Hurenkinder unterdrücken
\clubpenalty = 10000 
\widowpenalty = 10000 
\displaywidowpenalty = 10000 

\usepackage[utf8]{inputenc}
% Mathematischen Formelsatz erlauben
\usepackage{amsmath}
\usepackage{amssymb}
\usepackage[ngerman]{babel}

%% Wenn ein Index verwendet werden soll, folgendes auskommentieren
% \usepackage{makeidx}
% \makeindex

\usepackage{graphicx}

%% Literaturverzeichnis
\usepackage{jurabib}
\jurabibsetup{
	authorformat=abbrv,
        titleformat={short,commasep},
	commabeforerest,
	see,
        square,
}
% Typographischer Schnickschnack für Bibliographie
\renewcommand*{\jbauthorfont}{\textsc}
\renewcommand*{\biblnfont}{\scshape\textbf}
\renewcommand*{\bibfnfont}{\normalfont\textbf}


\usepackage{hyperref}
\hypersetup{ 
    colorlinks,
    citecolor=black,
    filecolor=black,
    linkcolor=black,
    urlcolor=black 
}

% Grad-Zeichen definieren
\newcommand{\grad}{\mbox{\(\mathsurround=0pt{}^\circ\)}}
% (r)-Zeichen definieren
\def\TReg{\textsuperscript{\textregistered}}
% z.B. Typographisch korrekt setzen
\usepackage{xspace}
\newcommand{\zB}{\mbox{z.\,B.}\xspace}
\newcommand{\dH}{\mbox{d.\,h.}\xspace}
\newcommand{\ua}{\mbox{u.\,a.}\xspace}

%% Schönere Tabellen
\usepackage{booktabs}
\usepackage{tabularx}

%% Ränder setzen
\usepackage[left=5cm,right=3cm,top=1.5cm,bottom=2cm,includeheadfoot]{geometry}

%% Abkürzungsverzeichnis
\usepackage[printonlyused]{acronym}

%% Gedrehte Tabellen
\usepackage{rotating}

%% Zeilenabstand 1.5. Nur setzen, wenn erforderlich, sieht ziemlich
%% hässlich aus.
% \usepackage{setspace} 
%% 150%
% \onehalfspacing % Zeilenabstand
%% 125%
%\linespread{1.25}
%% 110%
%\linespread{1.1}
%% 105%
%\linespread{1.05}
%% 102,5%
%\linespread{1.025}

%% Trennung tweaken: Manche Wörter will man ggf. anders trennen als
%% der Algorithmus das vorsieht oder ggf. auch garnicht
\hyphenation{Java-Script}
\hyphenation{Cookies}

\begin{document}

\begin{titlepage}

\includegraphics[width=8cm]{img/fhk.jpg}

\begin{center}
\textbf{\LARGE Fachhochschule Köln\\
Cologne University of Applied Sciences\\[0.5cm]
Campus Gummersbach\\[0.5cm]
Fakultät für Informatik und Ingenieurwissenschaften\\[1cm]}
 
\Large Verbundstudiengang Wirtschaftsinformatik\\[1cm]

\Large Template \\[1cm]
  
% Title
{ \huge \bfseries \LaTeX-Template für Thesen an der FH Köln}\\

\vfill

\begin{table}[h]
\centering
\begin{tabular}{p{4cm}p{5cm}}
  Prüfer:         & Prof. Dr. Max Mustermann \\
  Zweitprüfer:    & Prof. Dr. Monika Mustermann \\
                  &  \\
  vorgelegt am:   & \today \\
  von:            & Stephan Student \\
  aus:	  	  & Gummersbach \\
  Telefon-Nr.:	  & +49-123-456789 \\
  Matrikel-Nr.:   & 123456789 \\
  E-Mail-Adresse: & student@fh-koeln.de
\end{tabular}
\end{table}
\end{center}
\end{titlepage}

\tableofcontents

\chapter{Einleitung}

Dieses Template dient als Vorlage für Bachelor- und Masterthesen,
Haus-, Seminar- und Diplomarbeiten an der \ac{FH} Köln.

Dies ist ein Kapitel.

\section{Erster Beispielabschnitt}

Dies ist ein Abschnitt.

\subsection{Erster Beispielunterabschnitt}

Dies ist ein Unterabschnitt.

\subsubsection{Erster Unterunterabschnitt}

Dies ist ein Unterunterabschnitt.

\subsubsection{Zweiter Unterunterabschnitt}

Dies ist ein Unterunterabschnitt.

\subsection{Dritter Beispielunterabschnitt}

Dies ist ein Unterabschnitt.

\chapter*{Abkürzungsverzeichnis}
\addcontentsline{toc}{chapter}{Abkürzungsverzeichnis}
\begin{acronym}[Acronms]
\acro{FH}{Fachhochschule Köln}

\end{acronym}

\bibliography{thesis}{}
\bibliographystyle{jureco} %% jurabib, jureco, jurunsrt, jox

\listoffigures

\listoftables

\lstlistoflistings

%% \printindex

\appendix

\chapter{Erklärung}

Ich versichere, die von mir vorgelegte Arbeit selbständig verfasst zu
haben. Alle Stellen, die wörtlich oder sinngemäß aus veröffentlichten
oder nicht veröffentlichten Arbeiten anderer entnommen sind, habe ich
als entnommen kenntlich gemacht. Sämtliche Quellen und Hilfsmittel,
die ich für die Arbeit benutzt habe, sind angegeben. Die Arbeit hat
mit gleichem Inhalt bzw. in wesentlichen Teilen noch keiner anderen
Prüfungsbehörde vorgelegen.

\bigskip

Gummersbach, den \today

\bigskip

\bigskip

\bigskip

\bigskip

\bigskip

\bigskip

(Unterschrift)

\end{document}

%% end of file
